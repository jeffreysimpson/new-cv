%%%%%%%%%%%%%%%%%%%%%%%%%%%%%%%%%%%%%%%%%%%%%%%%%%%%%%%%%%%%%%%%%%%%%%%%
%%%%%%%%%%%%%%%%%%%%%% Simple LaTeX CV Template %%%%%%%%%%%%%%%%%%%%%%%%
%%%%%%%%%%%%%%%%%%%%%%%%%%%%%%%%%%%%%%%%%%%%%%%%%%%%%%%%%%%%%%%%%%%%%%%%

%%%%%%%%%%%%%%%%%%%%%%%%%%%%%%%%%%%%%%%%%%%%%%%%%%%%%%%%%%%%%%%%%%%%%%%%
%% NOTE: If you find that it says                                     %%
%%                                                                    %%
%%                           1 of ??                                  %%
%%                                                                    %%
%% at the bottom of your first page, this means that the AUX file     %%
%% was not available when you ran LaTeX on this source. Simply RERUN  %%
%% LaTeX to get the ``??'' replaced with the number of the last page  %%
%% of the document. The AUX file will be generated on the first run   %%
%% of LaTeX and used on the second run to fill in all of the          %%
%% references.                                                        %%
%%%%%%%%%%%%%%%%%%%%%%%%%%%%%%%%%%%%%%%%%%%%%%%%%%%%%%%%%%%%%%%%%%%%%%%%

%%%%%%%%%%%%%%%%%%%%%%%%%%%% Document Setup %%%%%%%%%%%%%%%%%%%%%%%%%%%%

% Don't like 10pt? Try 11pt or 12pt
\documentclass[10pt]{article}
\sloppy
\RequirePackage[T1]{fontenc}
% LaTeX will typeset using Computer Modern Roman, which a lot of
% non-mathematicians and non-engineers won't like. Also, a few PDF
% viewers may not render CMR very well. Instead, Times New Roman can
% be used. That's what this package does.
\usepackage{times}
\usepackage{fancyhdr,lastpage}
\usepackage{color,hyperref}
% This is a helpful package that puts math inside length specifications
\usepackage{calc}
% This package helps LaTeX auto-hyphenate hyphenated words if you use
% special hyphens. For example, bio\-/mimicry will properly hyphenate
% ``mimicry'' if necessary.
\usepackage[shortcuts]{extdash}
% Provides special list environments and macros to create new ones
\usepackage[shortlabels]{enumitem}

\hyphenation{citations}
\hyphenation{MNRAS}

% Layout: Puts the section titles on left side of page
\reversemarginpar

% Text formatting.
\newcommand{\foreign}[1]{\textit{#1}}
\newcommand{\etal}{\foreign{et~al.}}
\newcommand{\project}[1]{\textsl{#1}}
\definecolor{grey}{rgb}{0.5,0.5,0.5}
\newcommand{\deemph}[1]{\textcolor{grey}{\footnotesize{#1}}}

%
%         PAPER SIZE, PAGE NUMBER, AND DOCUMENT LAYOUT NOTES:
%
% The next \usepackage line changes the layout for CV style section
% headings as marginal notes. It also sets up the paper size as either
% letter or A4. By default, letter was used. If A4 paper is desired,
% comment out the letterpaper lines and uncomment the a4paper lines.
%
% As you can see, the margin widths and section title widths can be
% easily adjusted.
%
% ALSO: Notice that the includefoot option can be commented OUT in order
% to put the PAGE NUMBER *IN* the bottom margin. This will make the
% effective text area larger.
%
% IF YOU WISH TO REMOVE THE ``of LASTPAGE'' next to each page number,
% see the note about the +LP and -LP lines below. Comment out the +LP
% and uncomment the -LP.
%
% IF YOU WISH TO REMOVE PAGE NUMBERS, be sure that the includefoot line
% is uncommented and ALSO uncomment the \pagestyle{empty} a few lines
% below.
%

%% Use these lines for letter-sized paper
%\usepackage[paper=letterpaper,
%            %includefoot, % Uncomment to put page number above margin
%            marginparwidth=1.2in,     % Length of section titles
%            marginparsep=.05in,       % Space between titles and text
%            margin=1in,               % 1 inch margins
%            includemp]{geometry}

%% Use these lines for A4-sized paper
\usepackage[paper=a4paper,
            %includefoot, % Uncomment to put page number above margin
            marginparwidth=30.5mm,    % Length of section titles
            marginparsep=1.5mm,       % Space between titles and text
            margin=25mm,              % 25mm margins
            includemp]{geometry}

%% More layout: Get rid of indenting throughout entire document
\setlength{\parindent}{0in}

% Set up the custom unordered list.
\newcounter{refpubnum}
\newcommand{\cvlist}{}

\definecolor{numcolor}{rgb}{0.5,0.5,0.5}

\usepackage{xpatch}%
\newcommand\mnras{MNRAS}
\newcommand\mnrasl{MNRASL}
\newcommand\apjs{ApJS}
\newcommand\aj{AJ}                   % Astronomical Journal (the)
\newcommand{\doi}[2]{{\href{http://dx.doi.org/#1}{{#2}}}}
\newcommand{\ads}[2]{\href{http://adsabs.harvard.edu/abs/#1}{{#2}}}
\newcommand{\isbn}[1]{{\footnotesize(\textsc{isbn:}{#1})}}
\newcommand{\arxiv}[1]{{\href{http://arxiv.org/abs/#1}{arXiv:{#1}}}}


\pagestyle{fancy}
%\pagestyle{empty}      % Uncomment this to get rid of page numbers
\fancyhf{}\renewcommand{\headrulewidth}{0pt}
\fancyfootoffset{\marginparsep+\marginparwidth}
\newlength{\footpageshift}
\setlength{\footpageshift}
          {0.5\textwidth+0.5\marginparsep+0.5\marginparwidth-2in}
\lfoot{\hspace{\footpageshift}%
       \parbox{4in}{\, \hfill %
                    \arabic{page} of \protect\pageref*{LastPage} % +LP
%                    \arabic{page}                               % -LP
                    \hfill \,}}

% Finally, give us PDF bookmarks

\definecolor{darkblue}{rgb}{0.0,0.0,0.3}
\hypersetup{colorlinks,breaklinks,
            linkcolor=darkblue,urlcolor=darkblue,
            anchorcolor=darkblue,citecolor=darkblue}

%%%%%%%%%%%%%%%%%%%%%%%% End Document Setup %%%%%%%%%%%%%%%%%%%%%%%%%%%%


%%%%%%%%%%%%%%%%%%%%%%%%%%% Helper Commands %%%%%%%%%%%%%%%%%%%%%%%%%%%%

%%% HEADING AT TOP OF CURRICULUM VITAE

% The title (name) with a horizontal rule under it
% (optional argument typesets an object right-justified across from name
%  as well)
%
% Usage: \makeheading{name}
%        OR
%        \makeheading[right_object]{name}
%
% Place at top of document. It should be the first thing.
% If ``right_object'' is provided in the square-braced optional
% argument, it will be right justified on the same line as ``name'' at
% the top of the CV. For example:
%
%       \makeheading[\emph{Curriculum vitae}]{Your Name}
%
% will put an emphasized ``Curriculum vitae'' at the top of the document
% as a title. Likewise, a picture could be included:
%
%   \makeheading[{\includegraphics[height=1.5in]{my_picture}}]{Your Name}
%
% the picture will be flush right across from the name. For this example
% to work, make sure the extra set of curly braces is included. Also
% makes ure that \usepackage{graphicx} is somewhere in the preamble.
\newcommand{\makeheading}[2][]%
        {\hspace*{-\marginparsep minus \marginparwidth}%
         \begin{minipage}[t]{\textwidth+\marginparwidth+\marginparsep}%
             {\large \bfseries #2 \hfill #1}\\[-0.15\baselineskip]%
                 \rule{\columnwidth}{1pt}%
         \end{minipage}}

%%% SECTION HEADINGS

% The section headings. Flush left in small caps down pseudo-margin.
%
% Usage: \section{section name}
\renewcommand{\section}[1]{\pagebreak[3]%
    \vspace{1.3\baselineskip}%
    \phantomsection\addcontentsline{toc}{section}{#1}%
    \noindent\llap{\scshape\smash{\parbox[t]{\marginparwidth}{\hyphenpenalty=10000\raggedright #1}}}%
    \vspace{-\baselineskip}\par}

%%% LISTS

% This macro alters a list by removing some of the space that follows the list
% (is used by lists below)
\newcommand*\fixendlist[1]{%
    \expandafter\let\csname preFixEndListend#1\expandafter\endcsname\csname end#1\endcsname
    \expandafter\def\csname end#1\endcsname{\csname preFixEndListend#1\endcsname\vspace{-0.6\baselineskip}}}

% These macros help ensure that items in outer-type lists do not get
% separated from the next line by a page break
% (they are used by lists below)
\let\originalItem\item
\newcommand*\fixouterlist[1]{%
    \expandafter\let\csname preFixOuterList#1\expandafter\endcsname\csname #1\endcsname
    \expandafter\def\csname #1\endcsname{\let\oldItem\item\def\item{\pagebreak[2]\oldItem}\csname preFixOuterList#1\endcsname}
    \expandafter\let\csname preFixOuterListend#1\expandafter\endcsname\csname end#1\endcsname
    \expandafter\def\csname end#1\endcsname{\let\item\oldItem\csname preFixOuterListend#1\endcsname}}
\newcommand*\fixinnerlist[1]{%
    \expandafter\let\csname preFixInnerList#1\expandafter\endcsname\csname #1\endcsname
    \expandafter\def\csname #1\endcsname{\let\oldItem\item\let\item\originalItem\csname preFixInnerList#1\endcsname}
    \expandafter\let\csname preFixInnerListend#1\expandafter\endcsname\csname end#1\endcsname
    \expandafter\def\csname end#1\endcsname{\csname preFixInnerListend#1\endcsname\let\item\oldItem}}

% An itemize-style list with lots of space between items
%
% Usage:
%   \begin{outerlist}
%       \item ...    % (or \item[] for no bullet)
%   \end{outerlist}
\newlist{outerlist}{itemize}{3}
    \setlist[outerlist]{label=\enskip\textbullet,leftmargin=*}
    \fixendlist{outerlist}
    \fixouterlist{outerlist}

% An environment IDENTICAL to outerlist that has better pre-list spacing
% when used as the first thing in a \section
%
% Usage:
%   \begin{lonelist}
%       \item ...    % (or \item[] for no bullet)
%   \end{lonelist}
\newlist{lonelist}{itemize}{3}
    \setlist[lonelist]{label=\enskip\textbullet,leftmargin=*,partopsep=0pt,topsep=0pt}
    \fixendlist{lonelist}
    \fixouterlist{lonelist}

% An itemize-style list with little space between items
%
% Usage:
%   \begin{innerlist}
%       \item ...    % (or \item[] for no bullet)
%   \end{innerlist}
\newlist{innerlist}{itemize}{3}
    \setlist[innerlist]{label=\enskip\textbullet,leftmargin=*,parsep=0pt,itemsep=0pt,topsep=0pt,partopsep=0pt}
    \fixinnerlist{innerlist}

% An environment IDENTICAL to innerlist that has better pre-list spacing
% when used as the first thing in a \section
%
% Usage:
%   \begin{loneinnerlist}
%       \item ...    % (or \item[] for no bullet)
%   \end{loneinnerlist}
\newlist{loneinnerlist}{itemize}{3}
    \setlist[loneinnerlist]{label=\enskip\textbullet,leftmargin=*,parsep=0pt,itemsep=0pt,topsep=0pt,partopsep=0pt}
    \fixendlist{loneinnerlist}
    \fixinnerlist{loneinnerlist}

%%% EXTRA SPACE

% To add some paragraph space between lines.
% This also tells LaTeX to preferably break a page on one of these gaps
% if there is a needed pagebreak nearby.
\newcommand{\blankline}{\quad\pagebreak[3]}
\newcommand{\halfblankline}{\quad\vspace{-0.5\baselineskip}\pagebreak[3]}

%%% FORMATTING MACROS
\usepackage{url}

% You can adjust the style \url{} uses here:
% (options are: same, rm, sf, tt; defaults to tt)
\urlstyle{same}

% For \email{ADDRESS}, links ADDRESS to the url mailto:ADDRESS
% (uncomment to typeset the e\-/mail address in typewriter font;
%  otherwise, will be typeset in the \urlstyle above)
%\DeclareUrlCommand\emaillink{\urlstyle{tt}}
\providecommand*\emaillink[1]{\nolinkurl{#1}}
\providecommand*\email[1]{\href{mailto:#1}{\emaillink{#1}}}

\providecommand\BibTeX{{B\kern-.05em{\sc i\kern-.025em b}\kern-.08em \TeX}}


%%%%%%%%%%%%%%%%%%%%%%%% End Helper Commands %%%%%%%%%%%%%%%%%%%%%%%%%%%

%%%%%%%%%%%%%%%%%%%%%%%%% Begin CV Document %%%%%%%%%%%%%%%%%%%%%%%%%%%%

\begin{document}
\makeheading{Dr.~Jeffrey~D.~Simpson\\Curriculum vitae}

\section{Contact Information}

% NOTE: Mind where the & separators and \\ breaks are in the following
%       table. Table is one row made up of three parboxes. The left
%       parbox has address info, the middle parbox has a vertical bar,
%       and the right parbox has phone and electronic contact
%       information.
%
% MACROS: \rcollength is the width of the right column of the table
%             (adjust it to your liking; default is 1.85in).
%         \spacewidth is width of area between left and right boxes.
%
\newlength{\rcollength}\setlength{\rcollength}{2.3in}%
\newlength{\spacewidth}\setlength{\spacewidth}{10pt}
%
\begin{tabular}[t]{@{}p{\textwidth-\rcollength-\spacewidth}@{}p{\spacewidth}@{}p{\rcollength}}%

% Address box
\parbox{\textwidth-\rcollength-\spacewidth}{%
School of Physics\\
\href{http://www.unsw.edu.au/}{University of New South Wales}\\
Sydney NSW Australia}

&
% Uncomment to add a vertical bar in middle of contact information
% {\vrule width 0.5pt}
\parbox[m][5\baselineskip]{\spacewidth}{} &

% Non-snail-mail contact information
\parbox{\rcollength}{%
% (02) 9372 4838 \\
\email{jeffrey.simpson@unsw.edu.au}}
\end{tabular}

\section{Education}

\textbf{Ph.D. in Astronomy},
\href{http://www.canterbury.ac.nz/}{University of Canterbury}
New Zealand,
             \hfill 2014

\textbf{M.Sc. in Astronomy},
\href{http://www.canterbury.ac.nz/}{University of Canterbury}
New Zealand,
             \hfill 2009

\section{Current Employment}

\textbf{Post-Doctoral Research Fellow},
            University of New South Wales
            \hfill {2018}

\vspace{0.1in}

\section{Previous Employment}

\textbf{Research Fellow},
            Australian Astronomical Observatory
            \hfill {2015 to 2018}

\textbf{Research Fellow},
            Macquarie University
            \hfill {2013 to 2015}

\vspace{0.1in}

\section{Refereed Publications}
51 refereed publications. 12 refeered publications as first author.

Total citations~=~1530; h-index~=~21 (2021-01-05)
\begin{list}{}{\cvlist}
\item[{\color{numcolor}\scriptsize19}] Buder, Sven, Asplund, Martin, Duong, Ly, Kos, Janez, \etal\ (incl.\ \textbf{JDS}), 2018, \doi{10.1093/mnras/sty1281}{The GALAH Survey: second data release}, \mnras, \textbf{478}, 4513 (\arxiv{1804.06041}) [\href{http://adsabs.harvard.edu/abs/2018MNRAS.478.4513B}{8 citations}]

\item[{\color{numcolor}\scriptsize18}] \textbf{Simpson, Jeffrey D.}, 2018, \doi{10.1093/mnras/sty847}{The most metal-poor Galactic globular cluster: the first spectroscopic observations of ESO280-SC06}, \mnras, \textbf{477}, 4565 (\arxiv{1804.02606})

\item[{\color{numcolor}\scriptsize17}] Quillen, Alice C., De Silva, Gayandhi, Sharma, Sanjib, Hayden, Michael, \etal\ (incl.\ \textbf{JDS}), 2018, \doi{10.1093/mnras/sty865}{The GALAH survey: stellar streams and how stellar velocity distributions vary with Galactic longitude, hemisphere, and metallicity}, \mnras, \textbf{478}, 228 (\arxiv{1802.02924}) [\href{http://adsabs.harvard.edu/abs/2018MNRAS.478..228Q}{6 citations}]

\item[{\color{numcolor}\scriptsize16}] Duong, L., Freeman, K. C., Asplund, M., Casagrande, L., \etal\ (incl.\ \textbf{JDS}), 2018, \doi{10.1093/mnras/sty525}{The GALAH survey: properties of the Galactic disc(s) in the solar neighbourhood}, \mnras, \textbf{476}, 5216 (\arxiv{1801.01514}) [\href{http://adsabs.harvard.edu/abs/2018MNRAS.476.5216D}{8 citations}]

\item[{\color{numcolor}\scriptsize15}] Kos, Janez, Bland-Hawthorn, Joss, Freeman, Ken, Buder, Sven, \etal\ (incl.\ \textbf{JDS}), 2018, \doi{10.1093/mnras/stx2637}{The GALAH survey: chemical tagging of star clusters and new members in the Pleiades}, \mnras, \textbf{473}, 4612 (\arxiv{1709.00794}) [\href{http://adsabs.harvard.edu/abs/2018MNRAS.473.4612K}{8 citations}]

\item[{\color{numcolor}\scriptsize14}] Wittenmyer, Robert A., Sharma, Sanjib, Stello, Dennis, Buder, Sven, \etal\ (incl.\ \textbf{JDS}), 2018, \doi{10.3847/1538-3881/aaa3e4}{The K2-HERMES Survey. I. Planet-candidate Properties from K2 Campaigns 1--3}, \aj, \textbf{155}, 84 (\arxiv{1712.06774}) [\href{http://adsabs.harvard.edu/abs/2018AJ....155...84W}{7 citations}]

\item[{\color{numcolor}\scriptsize13}] Sharma, Sanjib, Stello, Dennis, Buder, Sven, Kos, Janez, \etal\ (incl.\ \textbf{JDS}), 2018, \doi{10.1093/mnras/stx2582}{The TESS-HERMES survey data release 1: high-resolution spectroscopy of the TESS southern continuous viewing zone}, \mnras, \textbf{473}, 2004 (\arxiv{1707.05753}) [\href{http://adsabs.harvard.edu/abs/2018MNRAS.473.2004S}{12 citations}]

\item[{\color{numcolor}\scriptsize12}] \textbf{Simpson, Jeffrey D.}, De Silva, Gayandhi, Martell, Sarah L., Navin, Colin A., \& Zucker, Daniel B., 2017, \doi{10.1093/mnras/stx2174}{ESO 452-SC11: the lowest mass globular cluster with a potential chemical inhomogeneity}, \mnras, \textbf{472}, 2856 (\arxiv{1708.06875}) [\href{http://adsabs.harvard.edu/abs/2017MNRAS.472.2856S}{4 citations}]

\item[{\color{numcolor}\scriptsize11}] \textbf{Simpson, Jeffrey D.}, De Silva, G. M., Martell, S. L., Zucker, D. B., \etal, 2017, \doi{10.1093/mnras/stx1892}{Siriusly, a newly identified intermediate-age Milky Way stellar cluster: a spectroscopic study of Gaia 1}, \mnras, \textbf{471}, 4087 (\arxiv{1703.03823}) [\href{http://adsabs.harvard.edu/abs/2017MNRAS.471.4087S}{7 citations}]

\item[{\color{numcolor}\scriptsize10}] Martell, S. L., Sharma, S., Buder, S., Duong, L., \etal\ (incl.\ \textbf{JDS}), 2017, \doi{10.1093/mnras/stw2835}{The GALAH survey: observational overview and Gaia DR1 companion}, \mnras, \textbf{465}, 3203 (\arxiv{1609.02822}) [\href{http://adsabs.harvard.edu/abs/2017MNRAS.465.3203M}{62 citations}]

\item[{\color{numcolor}\scriptsize9}] \textbf{Simpson, Jeffrey D.}, Martell, Sarah L., \& Navin, Colin A., 2017, \doi{10.1093/mnras/stw2781}{A broad perspective on multiple abundance populations in the globular cluster NGC 1851}, \mnras, \textbf{465}, 1123 (\arxiv{1610.08562}) [\href{http://adsabs.harvard.edu/abs/2017MNRAS.465.1123S}{7 citations}]

\item[{\color{numcolor}\scriptsize8}] Traven, G., Matijevi{\v{c}}, G., Zwitter, T., {\v{Z}}erjal, M., \etal\ (incl.\ \textbf{JDS}), 2017, \doi{10.3847/1538-4365/228/2/24}{The Galah Survey: Classification and Diagnostics with t-SNE Reduction of Spectral Information}, The Astrophysical Journal Supplement Series, \textbf{228}, 24 (\arxiv{1612.02242}) [\href{http://adsabs.harvard.edu/abs/2017ApJS..228...24T}{13 citations}]

\item[{\color{numcolor}\scriptsize7}] Kos, Janez, Lin, Jane, Zwitter, Toma{\v{z}}, {\v{Z}}erjal, Maru{\v{s}}ka, \etal\ (incl.\ \textbf{JDS}), 2017, \doi{10.1093/mnras/stw2064}{The GALAH survey: the data reduction pipeline}, \mnras, \textbf{464}, 1259 (\arxiv{1608.04391}) [\href{http://adsabs.harvard.edu/abs/2017MNRAS.464.1259K}{18 citations}]

\item[{\color{numcolor}\scriptsize6}] MacLean, B. T., Campbell, S. W., De Silva, G. M., Lattanzio, J., \etal\ (incl.\ \textbf{JDS}), 2016, \doi{10.1093/mnrasl/slw073}{An extreme paucity of second population AGB stars in the `normal' globular cluster M4}, \mnras, \textbf{460} (\arxiv{1604.05040}) [\href{http://adsabs.harvard.edu/abs/2016MNRAS.460L..69M}{21 citations}]

\item[{\color{numcolor}\scriptsize5}] \textbf{Simpson, Jeffrey D.}, De Silva, G. M., Bland-Hawthorn, J., Freeman, K. C., \etal, 2016, \doi{10.1093/mnras/stw746}{The GALAH survey: relative throughputs of the 2dF fibre positioner and the HERMES spectrograph from stellar targets}, \mnras, \textbf{459}, 1069 (\arxiv{1603.08991}) [\href{http://adsabs.harvard.edu/abs/2016MNRAS.459.1069S}{3 citations}]

\item[{\color{numcolor}\scriptsize4}] Sheinis, Andrew, Anguiano, Borja, Asplund, Martin, Bacigalupo, Carlos, \etal\ (incl.\ \textbf{JDS}), 2015, \doi{10.1117/1.JATIS.1.3.035002}{First light results from the High Efficiency and Resolution Multi-Element Spectrograph at the Anglo-Australian Telescope}, Journal of Astronomical Telescopes, Instruments, and Systems, \textbf{1}, 35002 [\href{http://adsabs.harvard.edu/abs/2015JATIS...1c5002S}{21 citations}]

\item[{\color{numcolor}\scriptsize3}] De Silva, G. M., Freeman, K. C., Bland-Hawthorn, J., Martell, S., \etal\ (incl.\ \textbf{JDS}), 2015, \doi{10.1093/mnras/stv327}{The GALAH survey: scientific motivation}, \mnras, \textbf{449}, 2604 (\arxiv{1502.04767}) [\href{http://adsabs.harvard.edu/abs/2015MNRAS.449.2604D}{174 citations}]

\item[{\color{numcolor}\scriptsize2}] \textbf{Simpson, Jeffrey D.}, \& Cottrell, P. L., 2013, \doi{10.1093/mnras/stt857}{Spectral matching for abundances of 848 stars of the giant branches of the globular cluster {\ensuremath{\omega}} Centauri}, \mnras, \textbf{433}, 1892 (\arxiv{1305.3059}) [\href{http://adsabs.harvard.edu/abs/2013MNRAS.433.1892S}{7 citations}]

\item[{\color{numcolor}\scriptsize1}] \textbf{Simpson, Jeffrey D.}, Cottrell, P. L., \& Worley, C. C., 2012, \doi{10.1111/j.1365-2966.2012.22012.x}{Spectral matching for abundances and clustering analysis of stars on the giant branches of {\ensuremath{\omega}} Centauri}, \mnras, \textbf{427}, 1153 (\arxiv{1209.0495}) [\href{http://adsabs.harvard.edu/abs/2012MNRAS.427.1153S}{9 citations}]
\end{list}

\section{Preprints}
\begin{list}{}{\cvlist}
\item[{\color{numcolor}\scriptsize13}] Zwitter, Toma{\v{z}}, Kos, Janez, Buder, Sven, \etal\ (incl.\ \textbf{JDS}), 2020, The GALAH+ Survey: A New Library of Observed Stellar Spectra Improves Radial Velocities and Reveals Motions within M67, arXiv e-prints (\arxiv{2012.12201})

\item[{\color{numcolor}\scriptsize12}] Buder, Sven, Sharma, Sanjib, Kos, Janez, \etal\ (incl.\ \textbf{JDS}), 2020, The GALAH+ Survey: Third Data Release, arXiv e-prints (\arxiv{2011.02505}) [\href{https://ui.adsabs.harvard.edu/#abs/2020arXiv201102505B}{17~citations}]

\item[{\color{numcolor}\scriptsize11}] Nandakumar, Govind, Hayden, Michael R., Sharma, Sanjib, \etal\ (incl.\ \textbf{JDS}), 2020, The GALAH survey: Milky Way disc metallicity and alpha-abundance trends in combined APOGEE-GALAH catalogues, arXiv e-prints (\arxiv{2011.02783}) [\href{https://ui.adsabs.harvard.edu/#abs/2020arXiv201102783N}{2~citations}]

\item[{\color{numcolor}\scriptsize10}] Spina, Lorenzo, Ting, Yuan-Sen, De Silva, Gayandhi M., \etal\ (incl.\ \textbf{JDS}), 2020, The GALAH survey: tracing the Galactic disk with Open Clusters, arXiv e-prints (\arxiv{2011.02533}) [\href{https://ui.adsabs.harvard.edu/#abs/2020arXiv201102533S}{2~citations}]

\item[{\color{numcolor}\scriptsize9}] Sharma, Sanjib, Hayden, Michael R., Bland-Hawthorn, Joss, \etal\ (incl.\ \textbf{JDS}), 2020, The GALAH Survey: Dependence of elemental abundances on age and metallicity for stars in the Galactic disc, arXiv e-prints (\arxiv{2011.13818}) [\href{https://ui.adsabs.harvard.edu/#abs/2020arXiv201113818S}{1~citation}]

\item[{\color{numcolor}\scriptsize8}] Hayden, Michael R., Sharma, Sanjib, Bland-Hawthorn, Joss, \etal\ (incl.\ \textbf{JDS}), 2020, The GALAH Survey: Chemical Clocks, arXiv e-prints (\arxiv{2011.13745}) [\href{https://ui.adsabs.harvard.edu/#abs/2020arXiv201113745H}{1~citation}]

\item[{\color{numcolor}\scriptsize7}] \textbf{Simpson, Jeffrey D.}, Martell, Sarah L., Buder, Sven, \etal, 2020, The GALAH Survey: Accreted stars also inhabit the Spite Plateau, arXiv e-prints (\arxiv{2011.02659}) [\href{https://ui.adsabs.harvard.edu/#abs/2020arXiv201102659S}{1~citation}]

\item[{\color{numcolor}\scriptsize6}] Kos, Janez, Bland-Hawthorn, Joss, Buder, Sven, \etal\ (incl.\ \textbf{JDS}), 2020, The GALAH survey: Chemical homogeneity of the Orion complex, arXiv e-prints (\arxiv{2011.02485})

\item[{\color{numcolor}\scriptsize5}] Clark, Jake T., Clerte, Mathieu, Hinkel, Natalie R., \etal\ (incl.\ \textbf{JDS}), 2020, The GALAH Survey: Using Galactic Archaeology to Refine our Knowledge of TESS Target Stars, arXiv e-prints (\arxiv{2008.05372}) [\href{https://ui.adsabs.harvard.edu/#abs/2020arXiv200805372C}{1~citation}]

\item[{\color{numcolor}\scriptsize4}] Li, Ting S., Koposov, Sergey E., Erkal, Denis, \etal\ (incl.\ \textbf{JDS}), 2020, Broken into Pieces: ATLAS and Aliqa Uma as One Single Stream, arXiv e-prints (\arxiv{2006.10763}) [\href{https://ui.adsabs.harvard.edu/#abs/2020arXiv200610763L}{8~citations}]

\item[{\color{numcolor}\scriptsize3}] Martell, Sarah, \textbf{Simpson, Jeffrey D.}, Balasubramaniam, Adithya, \etal, 2020, The GALAH survey: Lithium-rich giant stars require multiple formation channels, arXiv e-prints (\arxiv{2006.02106}) [\href{https://ui.adsabs.harvard.edu/#abs/2020arXiv200602106M}{8~citations}]

\item[{\color{numcolor}\scriptsize2}] Sharma, Sanjib, Hayden, Michael R., Bland-Hawthorn, Joss, \etal\ (incl.\ \textbf{JDS}), 2020, Fundamental relations for the velocity dispersion of stars in the Milky Way, arXiv e-prints (\arxiv{2004.06556}) [\href{https://ui.adsabs.harvard.edu/#abs/2020arXiv200406556S}{11~citations}]

\item[{\color{numcolor}\scriptsize1}] \textbf{Simpson, Jeffrey D.}, Stello, Dennis, Sharma, Sanjib, \etal, 2018, The GALAH and TESS-HERMES surveys: high-resolution spectroscopy of luminous supergiants in the Magellanic Clouds and Bridge, arXiv e-prints (\arxiv{1804.05900}) [\href{https://ui.adsabs.harvard.edu/#abs/2018arXiv180405900S}{1~citation}]
\end{list}

\section{Competitive observing proposals}

\textbf{Anglo-Australian Telescope}
\begin{innerlist}
\item[] Co-I: Dynamical Studies of DES Stellar Streams (10 nights) \hfill{18B}
\item[] Co-I: The HERMES-TESS program (8 nights) \hfill{18B}
\item[] Co-I: Open clusters with HERMES (5 nights) \hfill{18A}
\item[] Co-I: Open clusters with HERMES (13 nights) \hfill{17B}
\item[] Co-I: How Extended is the Stellar Envelope of NGC5694? (6 hours) \hfill{17A}
\item[] Co-I: The GALAH Survey: Phase 2 (35 nights/semester) \hfill{17A--17B}
\item[] Co-I: The HERMES K2-follow-up program \\ (12 nights/semester)  \hfill{16A--17B}
\item[] \textbf{PI}: Probing the low mass regime of globular clusters (6 hours) \hfill{16A}
\item[] Co-I: The GALAH Survey (35 nights/semester) \hfill{15A--16B}
\end{innerlist}
\vspace{0.1in}
\textbf{Keck Observatory}
\begin{innerlist}
\item[] \textbf{PI}: ESO452: Exploring self-enrichment in low mass stellar clusters\\ (0.5 nights)  \hfill{17A}
\end{innerlist}

% \section{Invited talks}
% \printbibliography[keyword={invitedtalk},heading=none,resetnumbers=true]
%
% \section{Contributed conference talks}
% \printbibliography[keyword={talk},heading=none,resetnumbers=true]
%
% \section{Conference posters}
% \printbibliography[keyword={poster},heading=none,resetnumbers=true]

\section{Conference Activity}
Organized the 2017 Southern Cross Astrophysics Conference on ``Surveying the Cosmos, The Science From Massively Multiplexed Surveys''

\section{Service To Profession}

Referee for articles in PASA and A\&A

\vspace{0.1in}

Referee for research funding proposal for Polish National Science Centre


\section{References Available to Contact}

\href
{https://rsaa.anu.edu.au/people/academics/dr-chris-lidman}
{\textbf{Chris Lidman}}
%
\begin{innerlist}
   \item \href{mailto:christopher.lidman@anu.edu.au}{christopher.lidman@anu.edu.au}
   \item (02) 6125 0238
   \item Research School of Astronomy \& Astrophysics Mount Stromlo Observatory Cotter Road Weston Creek, ACT 2611 Australia
\end{innerlist}

\halfblankline

\href
{https://rsaa.anu.edu.au/people/academics/dr-chris-lidman}
{\textbf{Gary Da Costa}}
\begin{innerlist}
  \item \href{mailto:gary.dacosta@anu.edu.au}{gary.dacosta@anu.edu.au}
  \item (02) 6125 8913
   \item Research School of Astronomy \& Astrophysics Mount Stromlo Observatory Cotter Road Weston Creek, ACT 2611 Australia
\end{innerlist}

\end{document}

%%%%%%%%%%%%%%%%%%%%%%%%%% End CV Document %%%%%%%%%%%%%%%%%%%%%%%%%%%%%

%----------------------------------------------------------------------%
% The following is copyright and licensing information for
% redistribution of this LaTeX source code; it also includes a liability
% statement. If this source code is not being redistributed to others,
% it may be omitted. It has no effect on the function of the above code.
%----------------------------------------------------------------------%
% Copyright (c) 2007, 2008, 2009, 2010, 2011 by Theodore P. Pavlic
%
% Unless otherwise expressly stated, this work is licensed under the
% Creative Commons Attribution-Noncommercial 3.0 United States License. To
% view a copy of this license, visit
% http://creativecommons.org/licenses/by-nc/3.0/us/ or send a letter to
% Creative Commons, 171 Second Street, Suite 300, San Francisco,
% California, 94105, USA.
%
% THE SOFTWARE IS PROVIDED "AS IS", WITHOUT WARRANTY OF ANY KIND, EXPRESS
% OR IMPLIED, INCLUDING BUT NOT LIMITED TO THE WARRANTIES OF
% MERCHANTABILITY, FITNESS FOR A PARTICULAR PURPOSE AND NONINFRINGEMENT.
% IN NO EVENT SHALL THE AUTHORS OR COPYRIGHT HOLDERS BE LIABLE FOR ANY
% CLAIM, DAMAGES OR OTHER LIABILITY, WHETHER IN AN ACTION OF CONTRACT,
% TORT OR OTHERWISE, ARISING FROM, OUT OF OR IN CONNECTION WITH THE
% SOFTWARE OR THE USE OR OTHER DEALINGS IN THE SOFTWARE.
%----------------------------------------------------------------------%
